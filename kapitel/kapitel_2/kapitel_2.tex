\newpage
\section{Kennzahlen}
Im letzten Kapitel wurde die Konversionspfad-Analyse (siehe Kapitel 2.1) erläutert und wie diese dazu beitragen kann, Ursachen für einen Ausstieg aus dem Online-Bestellprozess zu ermitteln. Dies ist aber nur möglich durch die Definition entsprechender Kennzahlen, die eine Aussage über die Performance der Schritte im Bestellprozess treffen können. Kennzahlen sind nach Peterson~\footcite[Vgl. ][Seite 8]{Peterson.2006} immer Verhältniszahlen und keine Rohdaten. In diesem Kapitel werden Kennzahlen betrachtet mit besonderem Bezug auf ihre Relevanz für die Optimierung und Erfolgsmessung eines E-Shops im B2C-Umfeld. Des Weiteren werden mögliche Handlungsempfehlungen aus der entsprechenden Kennzahl abgeleitet. Diese sollen zur Steigerung des Erfolgs eines E-Shops führen. 

\subsection{Look-to-Clickrate}
Diese Kennzahl misst das Verhältnis zwischen der Anzahl aller Besucher einer Produktseite und derer die daraufhin ein oder mehr Produkte in den Warenkorb gelegt haben. Die Look-to-Clickrate eignet sich daher für die Erfolgsmessung des ersten Zwischenschrittes in unserem Konversionspfad. Aktuell existiert keine deutsche Übersetzung für diese Kennzahl, daher wird der englische Begriff verwendet~\footcite[Vgl. ][Seite 99]{Hienerth.2010}. Hienerth~\footcite[Vgl. ][Seite 110]{Peterson.2010} sieht in dieser Kennzahl eine mögliche Aussage zur Produktbeliebtheit. Die Kennzahl sei jedoch nicht interessant, wenn der Wert unverändert bleibt. Es ist jedoch davon auszugehen, dass eine Schwachstelle in der Produktpräsentation nur durch die Look-to-Clickrate identifiziert werden kann.

\subsection{Abbruchrate}
Die Analyse der Zwischenschritte im Konversionspfad, ist relevant für die Einschätzung über den Erfolg eines E-Shops. Die Abbruchrate misst das Verhältnis der Absprünge vom Start des Warenkorbs bis hin zum Checkout-Prozess. Bei einer hohen Abbruchrate sollte der gesamte Kaufprozess von Anfang bis Ende genau betrachtet werden~\footcite[Vgl. ][Seite 377]{Kaushik.2007}. Diese Kennzahl kann somit als Auslöser für die Einleitung einer Konversionspfad-Analyse angesehen werden.

\subsection{Kaufabschlussrate}
Mit der Kaufabschlussrate wird das Verhältnis zwischen tatsächlichen Käufern und der gesamten Anzahl Besuchern gemessen. Kaushik~\footcite[Vgl. ][Seite 55-56]{Kaushik.2010} empfiehlt bei der Anzahl aller Besucher lediglich die einzelne Besucher (unique visitors) zu betrachten und nicht die Anzahl der Seitenaufrufe. In der Realität ist es häufig der Fall, dass ein Kunde die Preise auf verschiedenen Plattformen vergleicht und daher öfters eine Seite besucht und dann wieder verlässt bevor er sich für einen Kauf entschließt. Des Weiteren bewertet Kaushik~\footcite[Vgl. ][Seite 336-338]{Kaushik.2007} die Kaufabschlussrate als weniger bedeutungsvolle Kennzahl, da sich nur bedingt Handlungsempfehlungen ableiten lassen. Dennoch ist diese Kennzahl einer der wichtigsten wenn es darum geht den Erfolg eines E-Shops zu messen~\footcite[Vgl. ][Seite 31]{Peterson.2006}. Sie wird daher auch als Instrument für das Benchmarking verschiedener E-Shops verwendet. Zudem lässt sich mit dieser Kennzahl auch die Auswirkung einer Optimierungsmaßnahme überprüfen.

\subsection{Durschnittlicher Warenkorb}
Eine Kennzahl die eine Aussage über die Ausgabebereitschaft seiner Kunden liefert ist der Wert der durchschnittlichen Warenkörbe. Dieser Indikator misst das Verhältnis zwischen dem gesamten Umsatz dividiert durch die Anzahl Bestellungen. Kaushik~\footcite[Vgl. ][Seite 153]{Kaushik.2010} empfiehlt diese Kennzahl im Zusammenhang mit der Kaufabschlussrate zu betrachten, um eine Aussage über die Effektivität eines E-Shops treffen zu können. Eine hohe Kaufabschlussrate allein führt nicht automatisch zu einer Gewinnsteigerung, solange der durchschnittliche Warenkorbwert gering ist. Sinkt die Kaufabschlussrate und der durchschnittliche Warenkorbwert kann die Ursache in falsch adressiertes Werbung liegen~\footcite[Vgl. ][Seite 30-31]{Peterson.2006}. Demnach misst die Kennzahl, in Kombination mit der Kaufabschlussrate, den Erfolg eines E-Shops.

\subsection{R{\"u}cksendequote}
Die Rücksendung von Waren verursacht Kosten und wirkt sich dementsprechend stark auf den Umsatz eines E-Shops aus. Es entfällt nicht nur der Gewinn, dem man durch den Verkauf des Produktes erwirtschaftet hätte, insbesondere werden auch Kosten verursacht durch zusätzliche Aufwände im Bereich der Logistik, Administration und gegebenenfalls durch portofreie Rücksendungen~\footcite[Vgl. ][Seite 105]{Hienerth.2010}. Darüber hinaus entsteht auch ein Wertverlust durch die retoure des Produktes, welches folglich zu einer geringen Marge führt. Ziel eines jeden E-Shop sollte es sein das Verhältnis von Rücksendungen zu der Anzahl aller Bestellungen so gering wie möglich zu halten. Durch die Vermeidung von Retouren lässt sich nicht nur maßgeblich der Erfolg eines E-Shops beeinflussen, auch die Zufriedenheit der Kunden lässt sich somit messen~\footcite[Vgl. ]{Diwosch.2008}. Diese Kennzahl ermöglicht daher nicht nur die Optimierung des E-Shops, sondern kann auch als Messinstrument für den Erfolg angesehen werden.
