\newpage
\section{Fazit}
Mit dem Umsatz der durch E-Commerce in Deutschland umgesetzt wird steigt. Gleichzeitig verschärft sich der Markt durch zahlreiche Neueinsteiger und durch die erhöhte Preistransparenz beim Kunden. Um am Markt zu bestehen muss der eigene E-Shop kontinuierlich Optimiert werden. Das Web-Controlling im Zusammenhang mit den richtigen Kennzahlen muss dabei als zentraler Bestandteil der Strategie verstanden werden. Durch den Einsatz eines Web-Controlling Regelkreis, kann ein iterativer Prozess implementiert werden, der die kontinuierliche Optimierung des E-Shops unterstützt. Ein essentieller Bestandteil dieses Regelkreises ist die Analyse der erhobenen Kennzahlen. Der Aufwand für die Implementierung und Definition sollte dabei nicht unterschätzt werden. Die Definition der geeigneten Kennzahlen lässt sich anhand der Konversionspfad-Analyse ableiten. Bei der Definition ist darauf zu achten, dass Kennzahlen auch eine Handlung implizieren die gegebenenfalls zu einer Sofortmaßnahme führen. Die Konversionspfad-Analyse bildet den Prozess vom Besuch des E-Shops bis letztendlich zum Kaufabschluss und der Neugewinnung von Kunden ab. Durch die Analyse lassen sich Schwachstellen lokalisieren. Operative Kennzahlen liefern dabei Aussagen über die Performance der Zwischenschritt bis zum Kaufabschluss.
Schwierigkeiten entstehen, wenn zu viele Kennzahlen oder nicht zielorientierte Kennzahlen ohne Handlungsanweisung definiert wurden. Hierzu sollten vor der Definition bereits mögliche Schwachstellen im Prozess evaluiert sein. Diese Evaluierung unterstützt auch die Planung geeigneter Gegenmaßnahmen. Durch den richtigen Einsatz des Web-Controlling, kann einen Vorsprung gegenüber der Konkurrenz erarbeitet werden. E-Shops die heutzutage noch keinen oder nur einen geringen Aufwand im Bereich des Web-Controlling investieren, sollten sich zeitnah mit dieser Thematik auseinandersetzen. Andernfalls droht die Gefahr, Marktanteile gegenüber der Konkurrenz zu verlieren.
